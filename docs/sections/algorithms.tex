\section{Key Algorithms}
\label{sec:algorithms}
\index{algorithms}

% -----------------------------------------------------------------------
% Template for adding a new algorithm block:
%
% \subsection{<Algorithm Name>}
% \label{subsec:alg-<slug>}
% \index{algorithm!<slug>}
%
% Short description. Reference the source file and line numbers.
%
% \begin{equation}
%   \mathcal{L} = ...
% \end{equation}
% -----------------------------------------------------------------------

\subsection{Normalized Cross-Correlation (NCC) Loss}
\label{subsec:alg-ncc}
\index{algorithm!NCC loss}

Source: \texttt{model/correlator.py}, \texttt{siamese-ncc.py}.

Given transformed images $\hat{x}_1$ and $\hat{x}_2$ from the U-Net, the NCC score is:
\begin{equation}
  \mathrm{NCC}(\hat{x}_1, \hat{x}_2)
    = \frac{\sum_{i}(\hat{x}_{1,i} - \bar{x}_1)(\hat{x}_{2,i} - \bar{x}_2)}
           {\sqrt{\sum_i(\hat{x}_{1,i} - \bar{x}_1)^2 \cdot \sum_i(\hat{x}_{2,i} - \bar{x}_2)^2}}
\end{equation}

The training loss is MSE between the predicted NCC score and the binary label
$y \in \{0, 1\}$ (1 = matching pair, 0 = non-matching pair):
\begin{equation}
  \mathcal{L}_\text{NCC} = \bigl(\mathrm{NCC}(\hat{x}_1, \hat{x}_2) - y\bigr)^2
\end{equation}

\subsection{SIFT Combined Loss}
\label{subsec:alg-sift}
\index{algorithm!SIFT loss}

Source: \texttt{siamese-sift.py}, \texttt{model/kornia\_dog.py}, \texttt{model/kornia\_sift.py}.

The SIFT training loss is a weighted sum of two terms:
\begin{equation}
  \mathcal{L}_\text{SIFT} = \gamma \cdot \mathcal{L}_\text{detector}
                           + \zeta \cdot \mathcal{L}_\text{descriptor}
\end{equation}

where:
\begin{itemize}
  \item $\mathcal{L}_\text{detector}$ — DoG pyramid correlation loss encouraging
    both images to produce similar scale-space responses.
  \item $\mathcal{L}_\text{descriptor}$ — distance between SIFT descriptors extracted
    from corresponding keypoints.
  \item $\gamma$ — detector loss weight (CLI: \texttt{--gamma}, default 1).
  \item $\zeta$ — descriptor loss weight (CLI: \texttt{--zeta}, default 10).
\end{itemize}

\subsection{Image Tiling}
\label{subsec:alg-tiling}
\index{algorithm!tiling}

Source: \texttt{createTiledDataset.py}.

Large images are tiled into non-overlapping patches of fixed size
$(W_\text{crop}, H_\text{crop})$.
The overlap ratio parameter controls the stride between tiles:
\begin{equation}
  \text{stride} = \lfloor W_\text{crop} \cdot (1 - r_\text{overlap}) \rfloor
\end{equation}
where $r_\text{overlap} = 0$ gives non-overlapping tiles.

% Add new algorithm subsections here as features are implemented.
