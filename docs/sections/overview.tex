\section{Project Overview}
\label{sec:overview}

This repository contains the official training code for the \textbf{Science Robotics} paper:
\textit{``A seasonally invariant deep transform for visual terrain-relative navigation''}
(Fragoso et al., 2021).

The system trains deep neural networks (U-Net) to produce image transformations that are
invariant to seasonal appearance changes (e.g., summer vs.\ winter), enabling robust visual
terrain-relative navigation for robots.

\subsection{Scope Extension}

The project is being extended to support multi-modal image matching across:
\begin{itemize}
  \item Aerial thermal image to satellite image matching
  \item Aerial grayscale image to satellite image matching
  \item Frame-to-frame thermal video matching (temporal)
  \item Frame-to-frame visible video matching (temporal)
\end{itemize}

\subsection{Training Approaches}

Two training approaches are implemented:
\begin{description}
  \item[NCC-based (\texttt{siamese-ncc.py})] Optimizes for normalized cross-correlation
    registration. Loss is MSE between predicted NCC score and binary match label.
  \item[SIFT-based (\texttt{siamese-sift.py})] Optimizes for SIFT feature matching via a
    weighted combination of DoG pyramid correlation loss and SIFT descriptor distance loss:
    $\mathcal{L} = \gamma \cdot \mathcal{L}_\text{SIFT} + \zeta \cdot \mathcal{L}_\text{pyramid}$.
\end{description}

\subsection{Tech Stack}

\begin{tabular}{ll}
\toprule
\textbf{Component} & \textbf{Library / Version} \\
\midrule
Language        & Python 3.11 \\
Deep Learning   & PyTorch 1.11.0, PyTorch Lightning 1.7.7 \\
Computer Vision & Kornia 0.6.6, OpenCV 4.5.5.64 \\
Augmentation    & Albumentations 1.1.0 \\
Image I/O       & Pillow, imageio, rasterio \\
Visualization   & TensorBoard, matplotlib \\
Scientific      & NumPy, SciPy, scikit-image \\
Testing         & pytest \\
Linting         & ruff, black \\
\bottomrule
\end{tabular}

\subsection{Repository Layout}

\begin{lstlisting}
.
+-- siamese-ncc.py          # NCC training entry point
+-- siamese-sift.py         # SIFT training entry point
+-- siamese-inference.py    # Inference entry point
+-- createTiledDataset.py   # Preprocessing: crop images into tiles
+-- model/
|   +-- unet.py             # U-Net architecture
|   +-- unet_parts.py       # U-Net building blocks
|   +-- correlator.py       # Normalized cross-correlation layer
|   +-- kornia_dog.py       # DoG multi-scale pyramid
|   +-- kornia_sift.py      # SIFT descriptor extraction
+-- dataset/
|   +-- neg_dataset.py      # Siamese dataset for NCC training
|   +-- neg_sift_dataset.py # Siamese dataset for SIFT training
|   +-- inference_dataset.py
+-- utils/
|   +-- helper.py           # Normer, tensorboard, pyramid loss
+-- tests/                  # pytest test suite
+-- docs/                   # This documentation
+-- configs/                # YAML experiment configs (planned)
\end{lstlisting}

\subsection{Known Issues}
\label{subsec:known-issues}
\index{known issues}

\begin{description}
  \item[\texttt{siamese-sift.py:69}] \texttt{input1} and \texttt{input2} are both assigned
    from \texttt{data[0]}, so \texttt{data[1]} (the off-season image) is never used during
    SIFT training. Likely a bug.
  \item[\texttt{createTiledDataset.py:17}] \texttt{h, w, c = np.asarray(img).shape} assumes
    a 3-D array and crashes on grayscale PNGs loaded as 2-D.
  \item[\texttt{correlator.py:20}] Random noise (\texttt{torch.randn}) is added during
    normalization, introducing non-determinism at inference time.
  \item[Hardcoded normalization stats] Differ between NCC dataset
    ($\sigma=0.12/0.10$) and SIFT dataset ($\sigma=0.135/0.12$) with no documented rationale.
\end{description}
